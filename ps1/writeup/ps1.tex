\documentclass{pset}
\usepackage{multicol}
\usepackage[compact]{titlesec}


\newcommand{\TODO}[1][finish]{\textbf{TODO --- #1}}

\setlength\parindent{0em}
\setlength\parskip{1em}
\titlespacing*{\section}{0pt}{0pt}{0pt}

\hypersetup{
  colorlinks=true,
  urlcolor=cyan
}


\psnum{1}
\date{Thursday, February 6}
\versionnumber{1}

\begin{document}
\maketitle

\section*{Overview}

This assignment will take you through a sequence of small problems,
culminating in a final program that simulates a rock-paper-scissors
tournament.

\section*{Objectives}

\begin{itemize}
\item{} Gain familiarity with basic OCaml features such as lists,
  tuples, functions, pattern matching, and data types.
\item{} Practice writing programs in the functional style using
  immutable data, recursion, and higher-order functions.
\item{} Introduce the basic features of the OCaml type system.
\item{} Illustrate the impact of code style on readability,
  correctness, and maintainability.
\end{itemize}

\section*{Recommended reading}

The following supplementary materials may be helpful in completing
this assignment:
\begin{itemize}
  \item{} Lectures \href{http://www.cs.cornell.edu/Courses/cs3110/2014sp/lectures/1/lec01.pdf}{1}
    \href{http://www.cs.cornell.edu/Courses/cs3110/2014sp/lectures/2/lec02.html}{2}
    \href{http://www.cs.cornell.edu/Courses/cs3110/2014sp/lectures/3/lec03.html}{3}
\item{} Recitations 
  \href{http://www.cs.cornell.edu/Courses/cs3110/2014sp/recitations/1/rec01.html}{1}
  \href{http://www.cs.cornell.edu/Courses/cs3110/2014sp/recitations/2/tuples_records_data.html}{2}
  \href{http://www.cs.cornell.edu/Courses/cs3110/2014sp/recitations/3/rec03.html}{3}
\item{} \href{http://www.cs.cornell.edu/Courses/cs3110/2014sp/handouts/style.html}{The CS 3110 style guide}
\item{} \href{http://ocaml.org/learn/tutorials/}{The OCaml tutorial}
\item{} \href{https://realworldocaml.org/v1/en/html/index.html}{Real World OCaml, Chapters 1-3}
\end{itemize}

\section*{What to turn in}

Exercises marked [code] should be placed in \filename{ps1.ml} and will
be graded automatically. Exercises marked [written] should be placed
in \filename{written.txt} or \filename{written.pdf} and will be graded
by hand.

\newpage{}
\section*{Warm-up}

\exercise{} 
[written] Identify the types and values of the following expressions.
If the expressions are not well typed, briefly explain why not.

\begin{note}{Note}
Although the toplevel will give you the answers to these questions, we
recommend that you try them on your own before checking them against the
toplevel.  Figuring out types is a good skill to have for reading OCaml code
and for passing 3110 exams.
\end{note}

\begin{multicols}{2}
\begin{enumerate}[(a)]
\item{} \code{3 + 5}
\item{} \code{("zardoz", 3+5)}
\item{} \code{["zardoz"; 3+5]}
\columnbreak{}
\item{} \code{[List.hd] :: []}
\item{} \code{fun x y -> if x then y else x}
\item{} \code{fun a (b, c) -> a c (b c)}
% \item{} \code{3. +. 5.}
% \item{} \code{3 +. 5}
% \item{} \code{"zardoz"}
% \item{} \code{("zardoz", 3+5)}
% \columnbreak
% \item{} \code{[1; 2; 3]}
% \item{} \code{1::2::3::[]}
% \item{} \code{["zardoz"; 3+5]}
% \item{} \code{fun x -> 3 + x}
% \item{} \code{fun x -> [x]}
\end{enumerate}
\end{multicols}

\exercise{}
[written] Give expressions having the following types.
\begin{multicols}{2}
\begin{enumerate}[(a)]
\item{} \code{int -> int -> int}
\item{} \code{(int -> int) -> int}
\item{} \code{int -> (int -> int)}
\columnbreak{}
\item{} \code{'a -> 'a}
\item{} \code{'a -> 'b -> 'a}
\item{} \code{('a * 'b -> 'c) -> ('a -> 'b -> 'c)}
\end{enumerate}
\end{multicols}

\newpage{}
\section*{Code Style}

\exercise{}
\href{http://en.wikipedia.org/wiki/Rock-paper-scissors}{Rock-paper-scissors}
is a simple two-player game in which players independently choose one
of three symbols (rock, paper, or scissors), and reveal their choice
simultaneously. The winner of each round is selected using the
following ordering: rock beats scissors, scissors beat paper, and
paper beats rock. When the two players choose the same symbol, the
round ends in a draw.

The following function simulates a single round of a
rock-paper-scissors game, but is written with poor style.

\begin{ocaml}
(** 
 * [rps_round a b] accepts two symbols, [a] and [b], and determines the
 * winner.  "rock" beats "scissors", "scissors" beats "paper", and "paper"
 * beats "rock"
 *
 * @param a b One of the choices "rock", "paper" or "scissors"
 * @return One of { -1, 0, 1, 4 }. -1 means [a] won, 1 means [b] won, 
 *   0 indicates a tie.  4 indicates an invalid input.
 *)
let rps_round (a : string) (b : string) : int =
  if a = "rock" then (
    if b = "rock" then 0
    else if b = "paper" then 1
    else if b = "scissors" then -1
    else 4
  )
  else if a = "paper" then (
    if b = "rock" then -1
    else if b = "paper" then 0
    else if b = "scissors" then 1
    else 4
  )
  else if a = "scissors" then (
    if b = "rock" then 1
    else if b = "paper" then -1
    else if b = "scissors" then 0
    else 4
  )
  else 4
\end{ocaml}

\newpage{}
\begin{enumerate}[(a)]
\item{} [code] Modify \code{rps_round} using the following type definitions, 
      \begin{ocaml}
      type move   = Rock | Paper | Scissors
      type result = AWin | BWin  | Draw
      \end{ocaml}
      so that \code{a} and \code{b} have type \code{move}, and the
      function returns a value of type \code{result}.  The resulting
      function \code{rps_round_enum}.

      [written] Briefly explain (in one or two sentences) why this
      function is less error-prone than the version using
      \code{string} and \code{int}.

\item{} [code] Write a function \code{rps_round_nested_match}
  replacing each \code{if}-\code{then}-\code{else} expression with a
  \code{match} statement.

  [written] Explain in one or two sentences why \code{rps_round_nested_match} is less error-prone than
  \code{rps_round_enum}

\item{} [code] Simplify the function further in
  \code{rps_round_single_match} by replacing the entire body of the
  function with a single \code{match} statement (hint: match on a
  tuple).

\item{} [code] Write a function \code{rps_round_with_helper} that uses
  a helper function
    \begin{ocaml}
    beats : move -> move -> bool
    \end{ocaml}
  returning
  \code{true} if its first parameter wins against its second parameter
  and \code{false} otherwise. Use the catch-all pattern \code{_} in
  your implementation of \code{beats}.

 [written] Explain one way in which the \code{rps_round_with_helper}
 implementation is less error-prone than
 \code{rps_round_single_match}.  Explain one way in which
 \code{rps_round_with_helper} is \textit{more} error-prone than
 \code{rps_round_single_match} (Hint: consider the code changes that
 would be needed to add a fourth move \code{Nuke} that beats all of
 the other \code{move}s).
\end{enumerate}

\section*{OCaml programming}

\exercise{}
\begin{enumerate}[(a)]
\item{} [code] Write a function \code{all_pairs} that takes an
  \code{'a list} $\ell$ as input and returns a list containing all
  pairs of elements of $\ell$.  For example:
      \begin{ocaml}
# all_pairs ['a'; 'b'];;
- : (char * char) list = [('a', 'a'); ('a', 'b');
                          ('b', 'a'); ('b', 'b')]

# all_pairs [0; 1; 2];;
- : (int * int) list = [(0, 0); (0, 1); (0, 2); (1, 0);
                        (1, 1); (1, 2); (2, 0); (2, 1); (2, 2)]
      \end{ocaml}

   The order of the output list does not matter.

\item{} [code] Use \code{all_pairs} to implement a function
  \code{test_rps_eq} that takes two of your implementations of
  \code{rps_round} as arguments, and returns true if they agree on all
  inputs.  For example:
\begin{ocaml}
# test_rps_eq rps_round_enum rps_round_enum;;
- : bool = true

# test_rps_eq rps_round_enum rps_round_with_helper;;
- : bool = true

# let rps_round_bogus a b = AWin in
test_rps_eq rps_round_enum rps_round_bogus;;
- : bool = false

# let rps_round_bogus a b = AWin in
test_rps_eq rps_round_bogs rps_round_bogus;;
- : bool = true
\end{ocaml}

\item{} [code] Use \code{all_pairs} and \code{test_rps_eq} to
  implement a function \code{test_all_rps} that takes a list of
  \code{rps_round} implementations as arguments and returns true if
  all of them are equal on all inputs:
\begin{ocaml}
# test_all_rps [rps_round_enum; rps_round_single_match;
  rps_round_nested_match; rps_round_with_helper];;
- : bool = true

# let rps_round_bogus a b = AWin in
test_all_rps [rps_round_enum; rps_round_bogus];;
- : bool = false
\end{ocaml}
\end{enumerate}

\exercise{} A rock-paper-scissors player can be represented as a
function that takes in a list of the opponent's previous moves, and
returns a move for the next round:
\begin{ocaml}
type history = move list
type player  = history -> move
\end{ocaml}
For example, the player that always chooses \code{Rock} would be
represented as
\begin{ocaml}
let always_rock : player = fun history -> Rock
\end{ocaml}

By convention, we will store the most recent move at the head of the
list.  For example, in the following game,

\begin{center}
\begin{tabular}{rccccc}
round    &  1 & 2 & 3 & 4 & 5 \\
\hline
A's move & Rock & Paper & Paper & Scissors & ? \\
B's move & Rock & Paper & Paper & Scissors &   \\
\end{tabular}
\end{center}

the list passed to player A for the fifth round will be
\code{[Scissors; Paper; Paper; Rock]}.

\begin{enumerate}[(a)]
\item{} [code] Implement a player \code{beats_last} that always beats
  the most recent move that the opponent made.  \code{beats_last}
  should play Rock on the first move.  For example:

\begin{ocaml}
# beats_last [Rock; Paper; Paper; Scissors];;
- : move = Paper

# beats_last [Paper];;
- : move = Scissors

# beats_last [];;
- : move = Rock
\end{ocaml}

\item{} [code] Write a function \code{always_plays} that takes a \code{move} and returns
      a player that always plays that move.  For example:
\begin{ocaml}
# let always_paper = always_plays Paper;;
always_paper : player = <fun>

# always_paper [];;
- : move = Paper

# always_paper [Scissors; Scissors; Scissors];;
- : move = Paper
\end{ocaml}

\item{} [code] Write a function \code{rps_game} that takes two players
  and keeps running rounds until one of them wins.  It should return
  \code{true} if the first wins; false otherwise.  Use any of your
  implementations of \code{rps_round} from Exercise 1.
\end{enumerate}

\exercise{}
\begin{enumerate}[(a)]
\item{} [code] Write a function
% JNF: pair_merge seems like a better name.
\begin{ocaml}
pair_filter : ('a -> 'a -> 'a) -> 'a list -> 'a list
\end{ocaml}
\code{pair_filter} takes two inputs: a function \code{compare} that
returns one of two values, and a list \code{l} of values.  The
\code{pair_filter} function should break \code{l} into adjacent
pairs, and apply \code{compare} to each pair.  It should return a list
containing the resulting values.  For example,
\begin{ocaml}
pair_filter max [0; 5; 7; 2; 3; -10]
  $\downarrow$ [max 0 5; max 7 2; max 3 -10]
  $\downarrow$ [5; 7; 3]
\end{ocaml}

If the input list has odd length, then the last element should be
returned as well.  For example,
\begin{ocaml}
pair_filter min [1; 2; 3; 4; 5]
  $\downarrow$ [min 1 2; min 3 4; 5]
  $\downarrow$ [1; 3; 5]
\end{ocaml}

\item{} [code] Write a function
\begin{ocaml}
tournament : ('a -> 'a -> 'a) -> 'a list -> 'a option
\end{ocaml}
\code{tournament} should repeatedly call \code{pair_filter} on its input
list until only a single element \code{w} remains.  It should then return
\code{Some w}.  If the input list is empty, \code{tournament} should
return \code{None}.

For example,
\begin{ocaml}
tournament max [1; 4; 3; 2; 7; 9; 0]
  $\downarrow$ tournament max [4; 3; 9; 0]
  $\downarrow$ tournament max [4; 9]
  $\downarrow$ tournament max [9]
  $\downarrow$ Some 9
\end{ocaml}

\end{enumerate}

\section*{Comments}

[written] At the end of the file, please include any comments you have about
the problem set, or about your implementation.  This would be a good place to
document any extra Karma problems that you did (see below), to list any
problems with your submission that you weren't able to fix, or to give us
general feedback about the problem set.

\section*{Release files}

The accompanying release file \filename{ps1.zip} contains the following files:
\begin{itemize}
\item \filename{writeup/} contains this file and also the \filename{.tex} file
      we used to generate it (in case you are interested)
\item \filename{release/ps1.ml} and \filename{written.txt} are templates for
      you to fill in and submit.
\item \filename{ps1.mli} contains the interface and documentation for the
      functions that you will implement in \filename{ps1.ml}
\item \filename{examples.ml} contains all of the examples from this writeup in
      the form of unit tests.
\end{itemize}

\newpage{}
\section*{Karma Suggestions}

\begin{note}{Note}
We encourage you to think about different directions that you can take
the problem sets or different parts of OCaml or functional programming
that you are curious about.

You may submit any extra work you do in the Karma section of each
problem set.  We will comment on any extra work that you turn in, but
Karma \textbf{is completely optional and will not affect your grade in
any way}.
\end{note}

Here are some suggestions that may pique your interest:
\begin{itemize}
\item{} Implement a player that randomly chooses what symbol to play.
\item{} Implement a player that asks the user what to play.
\item{} The \code{game} function will often run forever (e.g. if a deterministic player
      plays against itself).  Write a function that runs a game but stops and
      returns \code{Draw} if the game runs for \code{k} rounds.
\item{} Write a function that takes a (deterministic) player and returns a new
      player that will beat the first player on the third round.
\item{} Write a best two of three, best three of five, or best $k$ of $(2k - 1)$
      tournament.
\item{} Use \code{tournament} and \code{rps_game} to implement a
  rock-paper-scissors tournament.  Find a way to easily identify the
  winner.
\item{} Use your code to implement a different game, such as tic-tac-toe.
\item{} Use \LaTeX with the 3110 style files for your written questions.
\end{itemize}

\end{document}

%%
%% vim: set et ai
%%
